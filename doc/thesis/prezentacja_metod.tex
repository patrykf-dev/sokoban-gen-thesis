\documentclass[sokoban_generation_thesis.tex]{subfiles}

W niniejszej pracy zdecydowano się zestawić cztery metody generowania poziomów \textit{Sokoban}, pogrupowanych w~tabeli \ref{tab:methods}. Dogłębna analiza metod proponowanych w~literaturze dla problemu generowania poziomów \textit{Sokoban} pozwoliła na~dobór reprezentatywnych metod. Wybrano dwie będące heurystykami, które nie korzystają z~technik uczenia maszynowego, oraz dwie -- korzystające. Ponadto, dobrano dwie, które opierają swoje działanie na~symulacji rozgrywki oraz dwie, które działają bez symulowania działań gracza.

Istotną zaletą analizowanych metod, które symulują rozgrywkę, jest możliwość podania przykładowej sekwencji ruchów, rozwiązujących daną planszę -- tej z~symulacji. Z~kolei metody, w~których nie symuluje się rozgrywki, mogą zostać zastosowane do~innych problemów niż generowanie plansz \textit{Sokoban}, ze~względu na~swoje uogólnienie analizowanego problemu.

\begin{table}[h!]
	\smallskip
	\centering
	\caption{Analizowane metody}
	\begin{tabular}{|l|l|l|}
		\hline
		& \textbf{Heurystyka} & \textbf{Uczenie maszynowe}\\ \hline
		\textbf{Symulacja rozgrywki} & SYM \cite{heur_sim} & MCTS \cite{sok_mcts} \\ \hline
		\textbf{Brak symulacji} & PDB \cite{heur_pdb} & PPO \cite{sok_ppo}\\
		\hline
	\end{tabular}
	\label{tab:methods}
\end{table}
