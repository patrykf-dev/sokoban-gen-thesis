\documentclass[a4paper,11pt,twoside]{report}
% KOMPILOWAĆ ZA~POMOCĄ pdfLaTeXa, PRZEZ XeLaTeXa MOŻE NIE BYĆ POLSKICH ZNAKÓW

\usepackage{subcaption}
\usepackage{subfiles}
\usepackage{afterpage}
\newcommand\blankpage{%
	\null
	\thispagestyle{empty}%
	\addtocounter{page}{-1}%
	\newpage}
% -------------- Kodowanie znaków, język polski -------------
\usepackage[OT4]{fontenc}
\usepackage[english,polish]{babel}

\usepackage{amsmath, amsfonts, amsthm, latexsym} % głównie symbole matematyczne, środowiska twierdzeń

\usepackage[final]{pdfpages} % inputowanie pdfa
%\usepackage[backend=bibtex, style=verbose-trad2]{biblatex}


% ---------------- Wczytywanie grafik ------------------

\usepackage{graphicx}
\graphicspath{{grafiki/}{../grafiki/}}


% ---------------- Tabele ------------------

\usepackage{float} % for "H" placement specifier

% ---------------------- Hiperłącza --------------------------

\usepackage{hyperref}
\PassOptionsToPackage{hyphens}{url}\usepackage{hyperref}

% ---------------- Marginesy, akapity, interlinia ------------------


\usepackage[inner=20mm, outer=20mm, bindingoffset=10mm, top=25mm, bottom=25mm]{geometry}


\linespread{1.5}
\allowdisplaybreaks

\usepackage{indentfirst} % opcjonalnie; pierwszy akapit z~wcięciem
\setlength{\parindent}{5mm}


% ---------------- Formatowanie pseudokodu  ------------------

\usepackage{xcolor}
\usepackage{listings}
\definecolor{codegreen}{rgb}{0,0.7,0}
\definecolor{codegray}{rgb}{0.5,0.5,0.5}
\definecolor{codepurple}{rgb}{0.58,0,0.82}
\definecolor{backcolour}{rgb}{0.95,0.95,0.92}
\lstdefinestyle{mystyle}{
	language=Python,
	deletekeywords={from},
	backgroundcolor=\color{backcolour},   
	commentstyle=\color{codegreen},
	keywordstyle=\color{magenta},
	numberstyle=\tiny\color{codegray},
	stringstyle=\color{codepurple},
	basicstyle=\ttfamily\footnotesize,
	breakatwhitespace=false,         
	breaklines=true,                 
	captionpos=b,                    
	keepspaces=true,                 
	numbers=left,                    
	numbersep=5pt,                  
	showspaces=false,                
	showstringspaces=false,
	showtabs=false,                  
	tabsize=4
}
\lstdefinestyle{mystylenonumbers}{
	language=Python,
	deletekeywords={from},
	backgroundcolor=\color{backcolour},   
	commentstyle=\color{codegreen},
	keywordstyle=\color{magenta},
	numberstyle=\tiny\color{codegray},
	stringstyle=\color{codepurple},
	basicstyle=\ttfamily\footnotesize,
	breakatwhitespace=false,         
	breaklines=true,                 
	captionpos=b,                    
	keepspaces=true,                 
	numbers=none,                    
	numbersep=5pt,                  
	showspaces=false,                
	showstringspaces=false,
	showtabs=false,                  
	tabsize=4
}

%--------------------------- ŻYWA PAGINA ------------------------

\usepackage{fancyhdr}
\pagestyle{fancy}
\fancyhf{}
% numery stron: lewa do~lewego, prawa do~prawego 
\fancyfoot[LE,RO]{\thepage} 
% prawa pagina: zawartość \rightmark do~lewego, wewnętrznego (marginesu) 
\fancyhead[LO]{\sc \nouppercase{\rightmark}}
% lewa pagina: zawartość \leftmark do~prawego, wewnętrznego (marginesu) 
\fancyhead[RE]{\sc \leftmark}

\renewcommand{\chaptermark}[1]{
\markboth{\thechapter.\ #1}{}}

% kreski oddzielające paginy (górną i~dolną):
\renewcommand{\headrulewidth}{0 pt} % 0 - nie ma, 0.5 - jest linia


\fancypagestyle{plain}{% to~definiuje wygląd pierwszej strony nowego rozdziału - obecnie tylko numeracja
  \fancyhf{}%
  \fancyfoot[LE,RO]{\thepage}%
  
  \renewcommand{\headrulewidth}{0pt}% Line at~the header invisible
  \renewcommand{\footrulewidth}{0.0pt}
}

% ---------------- Nagłówki rozdziałów ---------------------

\usepackage{titlesec}
\titleformat{\chapter}%[display]
  {\normalfont\Large \bfseries}
  {\thechapter.}{1ex}{\Large}

\titleformat{\section}
  {\normalfont\large\bfseries}
  {\thesection.}{1ex}{}
\titlespacing{\section}{0pt}{30pt}{20pt} 
%\titlespacing{\co}{akapit}{ile przed}{ile po} 
    
\titleformat{\subsection}
  {\normalfont \bfseries}
  {\thesubsection.}{1ex}{}


% ----------------------- Spis treści ---------------------------
\def\cleardoublepage{\clearpage\if@twoside
\ifodd\c@page\else\hbox{}\thispagestyle{empty}\newpage
\if@twocolumn\hbox{}\newpage\fi\fi\fi}


% kropki dla chapterów
\usepackage{etoolbox}
\makeatletter
\patchcmd{\l@chapter}
  {\hfil}
  {\leaders\hbox{\normalfont$\m@th\mkern \@dotsep mu\hbox{.}\mkern \@dotsep mu$}\hfill}
  {}{}
\makeatother

\usepackage{titletoc}
\makeatletter
\titlecontents{chapter}% <section-type>
  [0pt]% <left>
  {}% <above-code>
  {\bfseries \thecontentslabel.\quad}% <numbered-entry-format>
  {\bfseries}% <numberless-entry-format>
  {\bfseries\leaders\hbox{\normalfont$\m@th\mkern \@dotsep mu\hbox{.}\mkern \@dotsep mu$}\hfill\contentspage}% <filler-page-format>

\titlecontents{section}
  [1em]
  {}
  {\thecontentslabel.\quad}
  {}
  {\leaders\hbox{\normalfont$\m@th\mkern \@dotsep mu\hbox{.}\mkern \@dotsep mu$}\hfill\contentspage}

\titlecontents{subsection}
  [2em]
  {}
  {\thecontentslabel.\quad}
  {}
  {\leaders\hbox{\normalfont$\m@th\mkern \@dotsep mu\hbox{.}\mkern \@dotsep mu$}\hfill\contentspage}
\makeatother



% ---------------------- Spisy tabel i~obrazków ----------------------

\renewcommand*{\thetable}{\arabic{chapter}.\arabic{table}}
\renewcommand*{\thefigure}{\arabic{chapter}.\arabic{figure}}
\usepackage{caption}
\captionsetup[table]{name=Tabela}
%\let\c@table\c@figure % jeśli włączone, numeruje tabele i~obrazki razem


% --------------------- Definicje, twierdzenia etc. ---------------


\makeatletter
\newtheoremstyle{definition}%    % Name
{3ex}%                          % Space above
{3ex}%                          % Space below
{\upshape}%                      % Body font
{}%                              % Indent amount
{\bfseries}%                     % Theorem head font
{.}%                             % Punctuation after theorem head
{.5em}%                            % Space after theorem head, ' ', or~\newline
{\thmname{#1}\thmnumber{ #2}\thmnote{ (#3)}}%  % Theorem head spec (can be~left empty, meaning `normal')
\makeatother

% ----------------------------- POLSKI --------------------------------

\theoremstyle{definition}
\newtheorem{theorem}{Twierdzenie}[chapter]
\newtheorem{lemma}[theorem]{Lemat}
\newtheorem{example}[theorem]{Przykład}
\newtheorem{proposition}[theorem]{Stwierdzenie}
\newtheorem{corollary}[theorem]{Wniosek}
\newtheorem{definition}[theorem]{Definicja}
\newtheorem{remark}[theorem]{Uwaga}

% -------------------------- POCZĄTEK --------------------------


% --------------------- Ustawienia użytkownika ------------------

\newcommand{\tytul}{Zestawienie technik uczenia maszynowego i~heurystyk zastosowanych do~proceduralnego generowania poziomów w~grze Sokoban}
\newcommand{\tytulen}{Comparison of~machine learning techniques and heuristics used for procedural level generation in~Sokoban game}
\newcommand{\supervisor}{dr inż. Maciej Bartoszuk}



\begin{document}
\sloppy

\includepdf[pages=-]{titlepage}


% ------------------ STRONA Z~PODPISAMI AUTORA/AUTORÓW I~PROMOTORA ------------------


\thispagestyle{empty}\newpage
\null

\vfill

\begin{center}
\begin{tabular}[t]{ccc}

............................................. & \hspace*{100pt} & .............................................\\
podpis promotora & \hspace*{100pt} & podpis autora


\end{tabular}
\end{center}



% ---------------------------- ABSTRAKTY -----------------------------
% W~PRACY PO~POLSKU, NAPIERW STRESZCZENIE PL, POTEM ABSTRACT EN

{
\begin{abstract}

\begin{center}
\tytul
\end{center}

Celem pracy jest wszechstronne porównanie czterech technik, które zostaną zastosowane do~proceduralnego generowania poziomów w~grze \textit{Sokoban}. Zestawienie czterech fundamentalnie różnych technik pozwoli na~wyznaczenie ich mocnych i~słabych stron, dając tym samym podłoże do~dalszego rozwoju badań nad generowaniem proceduralnym. 

W pracy zostanie sprawdzone, jak konfigurowalne są~poszczególne metody pod kątem trudności i~rozmiaru poziomu. Zostanie zbadana efektywność czasowa każdej z~metod. Przy użyciu autorskich metryk zostanie również określone, jak skomplikowane poziomy mogą zostać wygenerowane przez prezentowane techniki.~\textit{Sokoban} jest prostą grą łamigłówkową, w~której gracz ma~za~zadanie przemieścić przedmioty w~wyznaczone miejsca na~planszy będącej prostokątną siatką, gdzie każde pojedyncze pole jest kwadratem. Trudność gry jest zależna od~rozmieszczenia przedmiotów i~przeszkód na~planszy (poziomie). Zastosowanie proceduralnego generowania poziomów ma~na~celu odciążyć twórcę gry z~wymyślania ich. Ponadto, generowanie proceduralne może potencjalnie tworzyć poziomy znacznie bardziej zawiłe niż wymyśliłby je~człowiek.

Większość opisywanych w~literaturze algorytmów wymaga czasochłonnego procesu uczenia agenta. Ten proces ma~charakter inkrementalny i~techniki oparte na~nim należą do~zbioru algorytmów uczenia maszynowego. Z~tego powodu, poza dwoma technikami z~tego działu, zostaną również zaprezentowane dwie heurystyki, co~pozwoli na~zestawienie podejścia heurystycznego z~uczeniem maszynowym. Ponadto, weryfikacji zostanie poddana wartość dodana płynąca z~wiedzy eksperckiej w~kontekście generowania poziomów.\\


\noindent \textbf{Słowa kluczowe:} Generowanie proceduralne, sztuczna inteligencja, heurystyka, \textit{Sokoban}, uczenie maszynowe
\end{abstract}
}

\null\thispagestyle{empty}\newpage

{\selectlanguage{english}
\begin{abstract}

\begin{center}
\tytulen
\end{center}

The purpose of~this paper is~to~comprehensively compare four techniques that are~applied to~procedural level generation in~the game \textit{Sokoban}. The comparison of~the four fundamentally different techniques allows to~determine their strengths and weaknesses, thus providing a~foundation for further research on~procedural generation. 

The thesis will examine how configurable each method is~in~terms of~difficulty and level size. The time efficiency of~each method will be~examined. Using author's metrics, it~is also determined how complex levels can be~generated by~the techniques presented.~\textit{Sokoban} is~a~simple puzzle game in~which the player has to~move objects to~designated locations on~a~board that is~a~rectangular grid where each individual box is~a~square. The difficulty of~the game depends on~the arrangement of~objects and obstacles on~the board (level). The use of~procedural generation of~levels is~intended to~ease developer's work. Furthermore, procedural generation has the potential to~create levels much more intricate than a~human would come up~with.

Most of~the algorithms described in~the literature require a~time-consuming process of~learning the agent. This process is~incremental and techniques based on~it~belong to~the set of~machine learning algorithms. For this reason, in~addition to~the two techniques in~this section, two heuristics will also be~presented to~contrast the heuristic approach with machine learning one. Furthermore, the value of~expert knowledge in~the context of~level generation will be~verified.\\

\noindent \textbf{Keywords:} Procedural content generation, artificial intelligence, heuristics, \textit{Sokoban}, machine learning
\end{abstract}
}


% --------------------- OŚWIADCZENIE -----------------------------------------


\null\thispagestyle{empty}\newpage

\null \hfill Warszawa, dnia ..................\\

\par\vspace{5cm}

\begin{center}
Oświadczenie
\end{center}

\indent Oświadczam, że~pracę magisterską pod tytułem ,,\tytul '', której promotorem jest \supervisor, wykonałem samodzielnie, co~poświadczam własnoręcznym podpisem.
\vspace{2cm}

\begin{flushright}
  \begin{minipage}{50mm}
    \begin{center}
      ..............................................

    \end{center}
  \end{minipage}
\end{flushright}

\thispagestyle{empty}
\newpage

\null\thispagestyle{empty}\newpage


% ------------------- 4. Spis treści ---------------------
\pagenumbering{gobble}
\tableofcontents
\thispagestyle{empty}

% -------------- 5. ZASADNICZA CZĘŚĆ PRACY --------------------
\null\thispagestyle{empty}\newpage
\pagestyle{fancy}
\pagenumbering{arabic}
\setcounter{page}{11} % JEŻELI Z~POWODU DUŻEJ ILOŚCI STRON W~SPISIE TREŚCI SIĘ NIE ZGADZA, TRZEBA ZMODYFIKOWAĆ RĘCZNIE

\chapter{Wstęp} \label{chap:wstep}
\subfile{wstep}

\chapter{Prezentacja metod} \label{chap:prezentacja_metod}
\subfile{prezentacja_metod}
\cleardoublepage
\section{Metoda SYM}
\subfile{prezentacja_metod_sym}
\cleardoublepage
\section{Metoda PDB}
\subfile{prezentacja_metod_pdb}
\cleardoublepage
\section{Metoda MCTS}
\subfile{prezentacja_metod_mcts}
\cleardoublepage
\section{Metoda PPO}
\subfile{prezentacja_metod_ppo}
\cleardoublepage

\chapter{Wyniki eksperymentów} \label{chap:work}
\subfile{wyniki_eksperymentow}
\cleardoublepage
\section{Metoda SYM} \label{subs:exp_sym}
\subfile{wyniki_eks_sym}
\cleardoublepage
\section{Metoda PDB} \label{subs:exp_pdb}
\subfile{wyniki_eks_pdb}
\cleardoublepage
\section{Metoda MCTS}
\subfile{wyniki_eks_mcts}
\cleardoublepage
\section{Metoda PPO} \label{subs:exp_ppo}
\subfile{wyniki_eks_ppo}
\cleardoublepage
\section{Ogólne metryki} \label{subs:metric_auth}
\subfile{ogolne_metryki}
\cleardoublepage
\section{Zestawienie metod}
\subfile{wyniki_eks_zestawienie}
\cleardoublepage
\chapter{Podsumowanie}
\subfile{podsumowanie}
\cleardoublepage


% -------------------- 6. Bibliografia -----------------------
% Bibliografia leksykograficznie wg~nazwisk autorów
\begin{thebibliography}{32}%jak ktoś ma~więcej książek, to~niech wpisze większą liczbę

\bibitem[1]{ucbv} Jean-Yves Audibert, Remi Munos, Csaba Szepesvári, \emph{Tuning Bandit Algorithms in~Stochastic Environments}, Algorithmic Learning Theory 18th International Conference, Sendai, Japan, October 1--4, 2007.

\bibitem[2]{spelunky_gen} Walaa Baghdadi, Fawzya Shams Eddin, Rawan Al-Omari, Zeina Alhalawani, Mohammad Shaker, Noor Shaker \emph{A Procedural Method for Automatic Generation of~Spelunky Levels}, European Conference on~the Applications of~Evolutionary Computation, 2015.
	
\bibitem[3]{heur_pdb} Damaris S.~Bento, Andre G.~Pereira and Levi H.~S.~Lelis, \emph{Procedural Generation of~Initial States of~Sokoban}, Federal University of~Rio Grande do~Sul, Federal University of~Viçosa, Brazil, 2019.

\bibitem[4]{pcgrl_scanning} Nathan Brewer, \emph{Computerized Dungeons and Randomly Generated Worlds: From Rogue to~Minecraft}, Proceedings of~the IEEE, vol. 105, no. 5, pp. 970--977, May 2017.

\bibitem[5]{pdb_original} Joseph C.~Culberson, Jonathan Schaeffer, \emph{Searching with pattern databases}, Canadian Conference on~Artificial Intelligence, pages 402–-416, 1996.

\bibitem[6]{sokoban_pspace} Joseph C.~Culberson, \emph{Sokoban is~PSPACE-complete}, 1997.

\bibitem[7]{massive_gen} Guy Davis, \emph{Massive - Multiple Agent Simulation System in~Virtual Environment}, 2003.
	
\bibitem[8]{simcity_neural} Sam Earle, \emph{Using fractal neural networks to~play simcity 1 and conways game of~life at~variable scales}, AIIDE Workshop on~Experimental AI~in~Games, 2019.
	
\bibitem[9]{pdb_additive} Ariel Felner, Richard E.~Korf, Sarit Hanan, \emph{Additive pattern database heuristics}, Journal of~Artificial Intelligence Research, vol. 22, pp.279-–318, 2004.

\bibitem[10]{galactic_arms_race} Erin J.~Hastings, Ratan K.~Guha, Kenneth O.~Stanley, \emph{Automatic Content Generation in~the Galactic Arms Race Video Game}, IEEE Transactions on~Computational Intelligence and AI~in~Games 1, vol. 4, pp. 245--263, 2010.

\bibitem[11]{mctsanalysis} Steven James, George Konidaris, Benjamin Rosman, \emph{An Analysis of~Monte Carlo Tree Search}, University of~the Witwatersrand, Johannesburg, South Africa.

\bibitem[12]{sok_metrics2} Petr Jarušek, Radek Pelánek, \emph{Difficulty rating of~sokoban puzzle}, STAIRS 2010, IOS Press, pp. 140--150, 2010.

\bibitem[13]{sok_metrics3} Petr Jarušek, Radek Pelánek, \emph{Human problem solving: Sokoban case study}, Faculty of~Informatics Masaryk University, Brno, 2010.

\bibitem[14]{sok_metrics} Petr Jarušek, Radek Pelánek, \emph{What Determines Difficulty of~Transport Puzzles?}, XXIV International FLAIRS Conference, Florida, USA, 2011.

\bibitem[15]{mcts_improve} Simon J.~Karman, ~\emph{Generating Sokoban Levels that are Interesting to~Play using Simulation}, Utrecht University, 2018.

\bibitem[16]{sok_mcts} Bilal Kartal, Nick Sohre, and Stephen J.~Guy, \emph{Data-Driven Sokoban Puzzle Generation with Monte Carlo Tree Search}, University of~Minnesota, United States, 2016.

\bibitem[17]{sok_ppo} Ahmed Khalifa, Philip Bontrager, Sam Earle, Julian Togelius, \emph{PCGRL: Procedural Content Generation via Reinforcement Learning}, New York University, United States, 2020.

\bibitem[18]{uct_original} Levente Kocsis, Csaba Szepesvári, \emph{Bandit based Monte-Carlo Planning}, European Conference on~Machine Learning, Berlin, Germany, September 18--22, 2006.

\bibitem[19]{pdb_novelty} Nir Lipovetzky, Hector Geffner, \emph{Best-first width search: Exploration and exploitation in~classical planning}, AAAI Conference on~Artificial Intelligence, pp. 3590–-3596, 2017.

\bibitem[20]{deep_learning_pcgrl} Jialin Liu, Sam Snodgrass, Ahmed Khalifa, Sebastian Risi, Georgios N.~Yannakakis, Julian Togelius, \emph{Deep learning for procedural content generation}, Neural Computing and Applications 33, pp. 19--37, 2021.

\bibitem[21]{ucbminimal} Francis Maes, Louis Wehenkel, Damien Ernst, \emph{Automatic Discovery of~Ranking Formulas for Playing with Multi-armed Bandits}, European Workshop on~Reinforcement Learning, Athens, Greece, September 9--11, 2011.

\bibitem[22]{solver_pdb} Andre G.~Pereira, Marcus Ritt, Luciana S.~Buriol, \emph{Optimal sokoban solving using pattern databases with specific domain knowledge}, Artificial Intelligence, vol. 227, pp. 52–-70, 2015.

\bibitem[23]{ale_deepq} Volodymyr Mnih, Kavukcuoglu K., Silver D., Rusu A.~A., \emph{Humanlevel control through deep reinforcement learning.} Nature 518(7540), 2015.

\bibitem[24]{pdb_example} Florian Pommerening, Gabriele Roger, Malte Helmert, \emph{Getting the most out of~pattern databases for classical planning}, International Joint Conference on~Artificial Intelligence, 2013.

\bibitem[25]{terragen} Che Mat Ruzinoor, Abdul Rashid Mohamed Shariff, Biswajeet Pradhan, Mahmud Rodzi Ahmad, \emph{A review on~3D terrain visualization of~GIS data: techniques and software}, Geo-spatial Information Science, vol. 15:2, pp. 105--115, 2012.

\bibitem[26]{game_industry_creative} Piotr Rykała, \emph{The growth of~the gaming industry in~the context of~creative industries}, Uniwersytet Ekonomiczny w~Katowicach, 2020.

\bibitem[27]{heur_worse} Joshua Taylor, Ian Parberry, \emph{Procedural generation of~Sokoban levels}, Technical Report LARC, Laboratory for Recreational Computing, University of~North Texas, 2011.

\bibitem[28]{mario_gan} Vanessa Volz, Jacob Schrum, Jialin Liu, Simon M.~Lucas, Adam Smith, Sebastian Risi, \emph{Evolving Mario Levels in~the Latent Space of~a~Deep Convolutional Generative Adversarial Network}, 2018.

\bibitem[29]{heur_sim} Wu~Yueyang, \emph{An Efficient Approach of~Sokoban Level Generation}, Graduate School of~Computer and Information Science, Hosei University, Japan, 2020.

\bibitem[30]{e3_game_industry} USA Today, \emph{E3 2021: Video games are bigger business than ever, topping movies and music combined}, 2021.

\bibitem[31]{sok_wiki} \textit{Sokoban} wiki: http://www.sokobano.de/wiki.

\bibitem[32]{jsoko_website} \textit{JSoko} website: https://www.sokoban-online.de.
\end{thebibliography}

\thispagestyle{empty}
\pagenumbering{gobble}

%\chapter*{Wykaz najważniejszych oznaczeń i~skrótów}
%\subfile{slowniczek}

% ----- 8. Spis rysunków - jeśli nie ma, zakomentować --------
\listoffigures
\thispagestyle{empty}


% ------------ 9. Spis tabel - jak wyżej ------------------
\renewcommand{\listtablename}{Spis tabel}
\listoftables
\thispagestyle{empty}


% 10. Spis załączników - jak nie ma~załączników, to~zakomentować lub usunąć

\chapter*{Spis załączników}
\begin{enumerate}
\item Płyta CD~zawierająca:
\begin{itemize}
	\item dokument z~treścią pracy dyplomowej,
	\item streszczenie w~języku polskim,
	\item streszczenie w~języku angielskim,
	\item kod programów.
\end{itemize}
\end{enumerate}
\thispagestyle{empty}


\end{document}
