\documentclass[sokoban_generation_thesis.tex]{subfiles}
Głównym celem pracy było dokładne zbadanie czterech metod generacji poziomów do~gry \textit{Sokoban}. Analizy i~wnioski z~badań przedstawionych w~rozdz.~\ref{chap:work} dają podłoże do~rozwoju prezentowanych metod. Autorskie metryki pozwoliły na~jednolite zestawienie metod, ukazując jak skomplikowane plansze można przy ich użyciu wygenerować. Dobór metod pozwolił na~zestawienie podejścia heurystycznego oraz opartego na~technikach uczenia maszynowego. Co~więcej, ukazano też wady i~zalety korzystania z~wiedzy eksperckiej w~kontekście generowania plansz.

Jak przedyskutowano w~p.~\ref{subs:solving_sokoban}, problem podania rozwiązania zadanej planszy jest skomplikowany obliczeniowo i~nie jest znany algorytm wielomianowy rozwiązujący go. Korzystając z~metod PPO i~PDB, należy wziąć pod uwagę, iż~nie zwracają one przykładowego rozwiązania planszy, w~przeciwieństwie do~SYM i~MCTS, co~przemawia na~ich korzyść. Większość tworzonych dziś aplikacji użytkowych, skupionych na~grach łamigłówkowych, prezentuje wraz ze~stawianymi wyzwaniami przykładowe rozwiązania.

Warto również pochylić się nad wydajnością czasową metod. W~sytuacji w~której dysponuje się odpowiednio dużymi zasobami pamięci, można każdą z~metod stosować równolegle, licząc na~poprawę jakości wynikowych plansz. Autorzy metod MCTS i~PDB prezentują właśnie taką skalowalną implementację. Metody SYM i~PPO również skorzystałyby na~współbieżnym uruchomieniu więcej niż jednej instancji algorytmu.

Metody które nie opierają swojego działania na~symulacji (PPO i~PDB), mogą zostać użyte do~zadań generowania plansz innych gier. W~przypadku pierwszej z~nich autorzy używają jej do~generowania plansz aż~pięciu innych gier niż \textit{Sokoban}. Z~kolei metoda PDB, która rozwiązuje zagadnienie przeszukiwania przestrzeni, mogłaby również z~powodzeniem zostać zaadaptowana do~innych problemów generacji. Ta~cecha obu metod istotnie przemawia na~ich korzyść, czyniąc je~bardziej uniwersalnymi.

Metodę SYM cechuje możliwość tworzenia plansz o~bardzo dużych rozmiarach, tym samym kreowania kształtów plansz. Metoda PDB wypada najlepiej w~kwestii tworzenia skomplikowanych poziomów średnich rozmiarów, jednak wymaga gotowego schematu planszy, na~którym rozmieści pudła. Metody PPO i~SYM radzą sobie jedynie w~tworzeniu średnio skomplikowanych plansz małych rozmiarów. W~ramach głębszej analizy problemu generowania plansz \textit{Sokoban}, warto rozważyć metodę hybrydową, która mogłaby przy pomocy metody SYM generować duże poziomy, dzielić je~na~podproblemy rozwiązywane metodami MCTS/PPO i~rozkładać finalne ułożenie pudeł przy pomocy metody PDB. Podobne podejście dzielenia problemu generacji na~mniejsze części było już wykorzystywane i~analizowane w~literaturze \cite{heur_worse}, jednak nie korzystało ono z~metod wyspecjalizowanych w~danych etapach generacji.

Każda z~metod opiera swoje działanie o~zestaw hiperparametrów, które należy starannie dobierać w~celu zmaksymalizowania jakości i~liczby generowanych plansz. Analizy przeprowadzone w~podrozdz.~\ref{subs:exp_sym}--\ref{subs:exp_ppo} jasno wykazały, że~ich dobór jest kluczowy. Wszystkie cztery opisane metody mają swoje zalety i~wady, specjalizując się w~różnych etapach i~zadaniach generowania plansz. 
