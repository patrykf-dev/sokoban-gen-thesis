\documentclass[sokoban_generation_thesis.tex]{subfiles}

Hiperparametry zestawianych metod podzielono na~3 grupy w~celu usystematyzowania podziału.

\begin{enumerate}
	\item Parametry problemu opisują żądane własności planszy.
	\item Parametry czasu opisują ograniczenia czasowe działania metody.
	\item Parametry pracy opisują działanie metody generującej.
\end{enumerate}

\begin{table}[h!]
	\smallskip
	\centering
	\begin{tabular}{|l|c|l|l|}
		\hline
		\textbf{Metoda} & 
		\begin{tabular}{@{}c@{}}\textbf{Parametry} \\ \textbf{planszy}\end{tabular} & 
		\begin{tabular}{@{}c@{}}\textbf{Parametry} \\ \textbf{czasu}\end{tabular} & 
		\begin{tabular}{@{}c@{}}\textbf{Parametry} \\ \textbf{pracy}\end{tabular} \\ \hline
		SYM & size, max box & f-it, b-it & boxs, pullds, pushds, routegen \\ \hline
		PDB & instance & time, runs & heur \\ \hline
		MCTS & size, max box & it~& heur \\ \hline
		PPO & size, max box & learning steps & repr, change perc, PPO hyper\\ \hline
	\end{tabular}
	\caption{Hiperparametry analizowanych metod}
	\label{tab:hyper_params}
\end{table}